\documentclass[conference]{IEEEtran}

\usepackage{mathptmx,amssymb,amsmath}
\usepackage{stmaryrd}
\usepackage{MnSymbol}
\usepackage{float}
\usepackage{amsthm}
\usepackage{array}
\usepackage[american]{babel}
\usepackage{enumerate}
\usepackage{fancyvrb}
\usepackage{cite}
%\usepackage{url}
\usepackage{mathpartir}
\usepackage{graphicx}
\usepackage{subfig}
\usepackage{color}
\usepackage{verbatim}
\usepackage{pifont}
\usepackage[lined,boxed,linesnumbered]{algorithm2e}

\widowpenalty=0
\clubpenalty=0
\displaywidowpenalty=0
\raggedbottom
\sloppy
\sloppypar

\topskip0pt
\parskip0pt
\partopsep0pt

\def\ind{\parindent}
\DefineVerbatimEnvironment{program}{Verbatim}
  {baselinestretch=1.0,xleftmargin=\ind,fontsize=\small,
   commandchars=\\\{\},samepage=true}
\DefineVerbatimEnvironment{programBox}{BVerbatim}
  {baselinestretch=1.0,xleftmargin=0pt,fontsize=\small,
   commandchars=\\\{\},samepage=true}

\def\denseitems{
    \itemsep1pt plus1pt minus1pt
    \parsep0pt plus0pt
    \parskip0pt\topsep0pt}

\def\exsep{1ex plus1ex}
\newcounter{exerc}
\newenvironment{exerc}%
{\refstepcounter{exerc}\vspace{\exsep}\noindent\hrulefill{}\nopagebreak\\%
 \noindent\textbf{Exercise \arabic{exerc}.}\ }
{\nopagebreak\par\vspace{-4pt}\noindent\hrulefill{}\vspace{\exsep}}

\newenvironment{exercSP}%
{\refstepcounter{exerc}\vspace{\exsep}
 \begin{samepage}\noindent\hrulefill{}\nopagebreak\\%
 \noindent\textbf{Exercise \arabic{exerc}.}\ }
{\nopagebreak\par\vspace{-4pt}\noindent\hrulefill{}
 \end{samepage}\vspace{\exsep}}


\def\defmacro#1{\expandafter\def\csname#1\endcsname}

\def\newdef#1#2{\expandafter\ifx\csname#1\endcsname\relax
    \defmacro{#1}{#2}%
    \else\message{Command "#1" already defined}\fi}

\newif\ifmore
\def\mapdef#1#2(#3){\def\args{#1:#2:#3,\end}%
    \moretrue
    \loop\expandafter\nextarg\args
         \ifmore\repeat}
\def\nextarg#1:#2:#3,#4\end{\def\next{#4}%
    \ifx\next\empty\morefalse
        \else\def\args{#1:#2:#4\end}\fi
    \newdef{#2#3}{#1{#3}}}
    
%%%
%%% Meta notes
%%%
\newcommand{\getRef}[1]{[\textbf{** #1}: \textit{refs?}]}
\newlength{\dummylen}
\newcommand{\NOTE}[1]{\setlength{\dummylen}{\fboxrule}\setlength{\fboxrule}{2pt}%
            \vspace{1ex}\noindent\hfill%
            \fbox{\begin{minipage}{.96\columnwidth}#1\end{minipage}}%
            \setlength{\fboxrule}{\dummylen}\hfill{}\vspace{1ex}}

\definecolor{red}{RGB}{255,0,0}
\definecolor{green}{RGB}{0,255,0}
\definecolor{purple}{RGB}{255,0,255}

\DefineVerbatimEnvironment{code}{Verbatim}{fontsize=\footnotesize,fontseries=b}


%%%
%%% MATH
%%%
\def\OB#1{\ifmmode#1\else\mbox{$#1$}\fi}
\newtheorem{theorem}{Theorem}
\newtheorem{lemma}{Lemma}
\newtheorem{corollary}{Corollary}
\newenvironment{definition}[1][Definition]{\begin{trivlist}
\item[\hskip \labelsep {\bfseries #1}]}{\end{trivlist}}


% shared CC definitions
% \newcommand{\OR}{\ |\ }
\newenvironment{fsyntax}{\hfill$\begin{array}{rcl@{\quad}l}}
                        {\end{array}$\hfill{\ }}
\newenvironment{syntax}{\[\begin{array}{rcl@{\quad}l}}
                       {\end{array}\]\ignorespacesafterend}
% \newenvironment{example}{\[\begin{array}{l}}
%                        {\end{array}\]\ignorespacesafterend}
\newenvironment{example}{\par\medskip\indent$\begin{array}{l}}
                       {\end{array}$\par\medskip\noindent\ignorespacesafterend}

%% overlay macros
%
\newlength{\overlaywidth}
\newlength{\overlayheight}

\newcommand{\voverlay}[3][-]{%
   \settowidth{\overlaywidth}{\mbox{\OB{#2}}}\divide\overlaywidth2%
   \hspace*{\overlaywidth}%
   \makebox[0mm]{\OB{#2}}%
   \if#1-\makebox[0mm]{\OB{#3}}\else%
         \raisebox{#1}[0mm][0mm]{\makebox[0mm]{\OB{#3}}}\fi%
   \hspace*{\overlaywidth}}

\newcommand{\vhoverlay}[3][-]{%
   \settowidth{\overlaywidth}{\mbox{\OB{#2}}}\divide\overlaywidth2%
   \settoheight{\overlayheight}{\mbox{\OB{#3}}}%
   \if#1-\else\addtolength{\overlayheight}{-#1}\fi
   \hspace*{\overlaywidth}%
   \makebox[0mm]{\OB{#2}}%
   \raisebox{-\overlayheight}[0mm][0mm]{\makebox[0mm]{\OB{#3}}}%
   \hspace*{\overlaywidth}}


%%
%% Math definitions
%%
\newtheorem{theorem}{Theorem}
\newtheorem{lemma}{Lemma}
\newtheorem{corollary}{Corollary}
\newenvironment{definition}[1][Definition]{\begin{trivlist}
\item[\hskip \labelsep {\bfseries #1}]}{\end{trivlist}}

%%% 
%%% MATH
%%%
\def\OB#1{\ifmmode#1\else\mbox{$#1$}\fi}
\newcommand{\vij}[3]{\OB{#1_{#2},\ldots,#1_{#3}}}
\newcommand{\vi}[2]{\vij{#1}{1}{#2}}
\newcommand{\vn}[2][n]{\vi{#2}{#1}}
\newcommand{\vnButi}[3][n]{\OB{#2_1,\ldots,#2_{#3-1},#2_{#3+1},\ldots,#2_{#1}}}
\newcommand{\set}[1]{\OB{\{#1\}}}
\newcommand{\vect}[1]{\OB{\langle#1\rangle}}
\newcommand{\cseq}[2][n]{\OB{^{#2:1..#1}}}
\newcommand{\iPat}[2][i]{\OB{[#1\mathop:#2]}}

% \newcommand{\xn}[1][n]{\OB{^{\bar{#1}}}}
% \newcommand{\xn}[1][n]{\OB{_{\bar{#1}}}}

\newcommand{\dom}[1]{\OB{\textit{dom}(#1)}}
\newcommand{\rng}[1]{\OB{\textit{rng}(#1)}}

\newcommand{\semL}{\OB{[\![}}
\newcommand{\semR}{\OB{]\!]}}
\newcommand{\sem}[2][{}]{\OB{\semL#2\semR_{#1}}}
\newcommand{\subst}[3]{\OB{[#1/#2]#3}}
%% multi-line definitions
\newcommand{\Begindef}{\ =\ \BegindefNES}
\newcommand{\BegindefNES}{\left\{\begin{array}{@{}l@{\quad}l}}
\newcommand{\Enddef}{\end{array}\right.}
% \newcommand{\If}{\textrm{if\ }}
\newcommand{\Otherwise}{\textrm{otherwise}}

\newcommand{\FV}[1]{\OB{\textit{FV}(#1)}}
\newcommand{\FD}[1]{\OB{\textit{FD}(#1)}}

% \newcommand{\bigstep}[3][\Delta]{\OB{#1 :- #2\Downarrow#3}}

\newcommand{\mrk}[1]{\underline{#1}}

%% 
%%  Choice calculus syntax
%%

%%% Documents and trees
% \newcommand{\prog}[1]{\texttt{#1}}
% \newcommand{\sub}[1]{\OB{{\scriptstyle\langle}#1{\scriptstyle\rangle}}}
\newcommand{\sub}[1]{\OB{\mathord\Yleft#1\mathord\Yright}}
\newcommand{\tr}[2][a]{\OB{#1\sub{#2}}}
\newcommand{\trn}[2][a]{\OB{#1\sub{\vn{#2}}}}
% \newcommand{\trPat}[4][a]{\tr[#1]{#2_{#3}|#4}}

%%% Key words & formatting/indenting
\newcommand{\CCkeyw}[1]{\textbf{\textrm{#1}}}
\newcommand{\LET}{\CCkeyw{let}}
\newcommand{\DIM}{\CCkeyw{dim}}
\newcommand{\IN}{\CCkeyw{in}}
\newcommand{\Ind}[2][{}]{\hphantom{#1}\makebox[0mm][r]{#2}\ }

%%% Choices
\newcommand{\chcL}{\langle}
\newcommand{\chcR}{\rangle}
\newcommand{\chc}[2][D]{\OB{#1\chcL#2\chcR}}
% \newcommand{\chcPP}[3][D]{\chc[#1]{\prog{#2},\prog{#3}}}

%%% Dimensions
\newcommand{\Dim}[2][D]{\OB{\DIM\ \chc[#1]{#2}}}
\newcommand{\DimIn}[3][D]{\Dim[#1]{#2}\ \IN\ #3}

%%% Let expressions
% \newcommand{\Let}[2]{\OB{\LET\ #1 = #2}\ }
% \newcommand{\Let}[3][-]{\OB{\if#1-\LET\ \else\Ind[\LET]{#1}\fi#2 = #3}}
% \newcommand{\Let}[3][\LET]{\OB{\Ind[#1]{\LET}#2 \texttt{=} #3}}
\newcommand{\Let}[3][\LET]{\OB{\Ind[#1]{\LET}#2 {\small=} #3}}
\newcommand{\LetIn}[3]{\Let{#1}{#2}\ \IN\ #3}


%%% OLD Choice notation / abbreviation
% \newcommand{\ptch}[2][L]{\OB{#2^{#1}}}
% \newcommand{\ptch}[2][L]{\OB{\overline{#2}^{#1}}}
% \newcommand{\ptch}[2][L]{\OB{#1\mathord:\ #2}}
% \newcommand{\ptch}[2][L]{\OB{#1\mathord|\ #2}}
\newcommand{\ptch}[2][L]{\OB{#1\mapsto #2}}
% \newcommand{\ptch}[2][L]{\OB{#1\to #2}}
% \newcommand{\ptch}[2][L]{\OB{#1\mathord\to #2}}
\newcommand{\ochc}[1]{\OB{\set{#1}}}
\newcommand{\chcn}[3][n]{\OB{\chc{\ptch[#2_1]{#3_1},\ldots,\ptch[#2_#1]{#3_#1}}}}
\newcommand{\ochcA}[2]{\ochc{\ptch[#1]{#2}}}
\newcommand{\ochcB}[4]{\ochc{\ptch[#1]{#2},\ptch[#3]{#4}}}
\newcommand{\ochcBi}[4]{\ochc{\ptch[\textit{#1}]{#2},\ptch[\textit{#3}]{#4}}}
\newcommand{\ochcC}[6]{\ochc{\ptch[#1]{#2},\ptch[#3]{#4},\ptch[#5]{#6}}}
\newcommand{\ochcD}[8]{\ochc{\ptch[#1]{#2},\ptch[#3]{#4},\ptch[#5]{#6},\ptch[#7]{#8}}}
%%% OLD Choice binding
\newcommand{\lbnd}[2]{\OB{#1=#2}}
\newcommand{\bnd}[2]{\OB{#1\leftarrow#2}}
\newcommand{\cbnd}[2]{\bnd{#1}{\chc{#2}}}
\newcommand{\cbind}[2]{\bnd{#1}{\chc{#2}}}
\newcommand{\bind}[3]{\OB{#1:\cbind{#2}{#3}}}
\newcommand{\bindB}[5]{\OB{#1:\cbind{#2}{#3};\cbind{#4}{#5}}}
\newcommand{\bindk}[3]{\OB{#1:\cbind{#2_1}{#3_1};\ldots;\cbind{#2_k}{#3_k}}}
\newcommand{\bindkC}[3]{\OB{#1:\bnd{#2_1}{#3_1};\ldots;\bnd{#2_k}{#3_k}}}
\newcommand{\noDbindkC}[2]{\OB{\bnd{#1_1}{#2_1};\ldots;\bnd{#1_k}{#2_k}}}
\newcommand{\noCode}{\OB{\bullet}}
\newcommand{\tagsName}{\textit{tags}}
\newcommand{\tags}[1]{\OB{\tagsName(#1)}}
\newcommand{\ctags}[1]{\OB{\textit{ctags}(#1)}}
\newcommand{\ddom}[1]{\OB{\textit{dom}^*(#1)}}

%% formatting choice names/ids
% \newcommand{\choice}[1]{\OB{{\cal #1}}}

%% names for change objects
% \newcommand{\chObj}{\OB{\omega}}
% \newcommand{\chObjB}{\OB{\omega'}}
% \newcommand{\chObj}{\OB{\bar{O}}}
% \newcommand{\chObjB}{\OB{\bar{O'}}}
% \newcommand{\vo}{\OB{\bar{O}}}
% \newcommand{\vo}{\OB{V}}
% \newcommand{\voB}{\OB{V'}}


%%
%% Static analysis
%%
\newcommand{\wdim}[2][\Delta]{\OB{#1\vdash#2}}

%%% Context notation
% \newcommand{\ctx}[1]{\OB{^{\sswarrow}C[#1]}}
%\newcommand{\ctx}[1]{\OB{\hat{C}[#1]}}
\newcommand{\ctx}[2][C]{\OB{#1[#2]}}


%%% Tag selection
\newcommand{\sel}[2][D.t]{\OB{#2\mathop\shortuparrow#1}}
% \newcommand{\tsel}[3][\Delta]{\OB{#2|_{#1}^{#3}}}
\newcommand{\tsel}[2][s]{\OB{\lfloor#2\rfloor_{#1}}}
% \newcommand{\tsel}[3][\Delta]{\OB{\sigma(#1,#2,#3)}}
% \newcommand{\tseltd}[2][\Delta]{\tsel[#1]{#2}{d.t}}
% \newcommand{\tselq}[2][\Delta]{\tsel[#1]{#2}{q}}

\newcommand{\dimSym}{\OB{\delta}}
\newcommand{\dims}[1]{\dimSym(#1)}
\newcommand{\dimProd}{\OB{\otimes}}
\newcommand{\dimSum}{\OB{\oplus}}
\newcommand{\dimUnit}{\textbf{1}}

\newcommand{\expSym}{\OB{\mu}}
\newcommand{\expEnv}{\OB{\rho}}
\newcommand{\expand}[2][\expEnv]{\expSym_{#1}(#2)}


\newcommand{\chcdep}[2]{\OB{#1\leftarrow#2}}

%% name of positional relationship
% \newcommand{\prel}{\OB{\Delta}}
% 
% \newcommand{\mkchc}[3][p]{\OB{#2[#1]\mathord{:}#3}}
% \newcommand{\addchc}[3][p]{\OB{#2[#1]\mathord{:}\mathord\cup#3}}
% \newcommand{\chcex}[2][p]{\OB{#2_{#1}}}

%% 
%%  Choice calculus semantics
%%

\newcommand{\tselB}[3]{\tsel[#3]{\tsel[#2]{#1}}}
\newcommand{\lft}[1]{\OB{\mathord\uparrow#1}}
\newcommand{\plain}[1]{\OB{\underline{#1}}}
\newcommand{\allsel}[2]{\OB{#1\mathord\Downarrow#2}}
% \newcommand{\lsel}[2][t_1,\ldots,t_n]{\OB{#2.\set{#1}}}

%%% Dimensions
\newcommand{\choicesSym}[1][{}]{\OB{\Gamma^{#1}}}
\newcommand{\choices}[2][{}]{\OB{\choicesSym[#1](#2)}}
\newcommand{\dimm}[1]{\dimSym(#1)}

%%% Variations
\newcommand{\variSym}{\OB{V}}
\newcommand{\vari}[1]{\OB{\variSym(#1)}}


%% 
%%  Design theory
%%
\newcommand{\equivSym}{\OB{\sim}}

\newcommand{\equalt}[3][C]{\OB{#2\equivSym_{#1}#3}}
\newcommand{\equtag}[3][C]{\OB{#2\equivSym_{#1}#3}}

% \newcommand{\dropName}{\textit{rem}}
% \newcommand{\drop}[4][C]{\OB{\dropName\textit{#2}_{#1}^{#3}(#4)}}
% \newcommand{\dropA}[3][C]{\drop{A}{#2}{#3}}
% \newcommand{\dropT}[3][C]{\drop{T}{#2}{#3}}
% \newcommand{\dropA}[3][C]{\OB{\alpha_{#1}^{#2}(#3)}}
% \newcommand{\dropA}[3][C]{\OB{\alpha_{#2}^{#1}(#3)}}


\newcommand{\dropi}[4][C]{\OB{\bar{#2}_{#1/#3}(#4)}}
\newcommand{\drop}[3][C]{\OB{\bar{#2}_{#1}(#3)}}

\newcommand{\dropA}[3][C]{\dropi[#1]{\alpha}{#2}{#3}}
\newcommand{\dropT}[3][C]{\dropi[#1]{\tau}{#2}{#3}}
\newcommand{\dropC}[2][C]{\drop[#1]{\gamma}{#2}}
\newcommand{\dropD}[2][C]{\drop[#1]{\delta}{#2}}



\newcommand{\vequiv}[2]{\OB{#1\sim#2}}
\newcommand{\tequiv}[4][e]{\OB{#3\equivSym^{#1}_{#2}#4}}
\newcommand{\tsubst}[4][D]{\OB{[#1:#2/#3]#4}}




%%% Properties of dimensions
% \newcommand{\indep}[2]{\OB{#1\rightleftarrows#2}}
% \newcommand{\depnd}[2]{\OB{#1\Leftarrow#2}}
\newcommand{\indep}[2]{\OB{#1\mathop\|#2}}
\newcommand{\related}[2]{\OB{#1\mathop\sim#2}}
\newcommand{\relclSym}{\OB{\vhoverlay[1.6ex]{\sim}{\scriptstyle *}}}
% \newcommand{\relclSym}{\OB{\overset{*}{\sim}}}
\newcommand{\relcl}[2]{\OB{#1\,\relclSym\,#2}}
\newcommand{\depnd}[2]{\OB{#1\rightarrow#2}}
\newcommand{\synch}[2]{\OB{#1\leftrightarrow#2}}
\newcommand{\ovlap}[2]{\OB{#1\leftrightharpoons#2}}


%%% Operations on versioned objects
\newcommand{\factorSym}{\OB{\varphi}}
\newcommand{\factor}[1]{\OB{\factorSym(#1)}}
\newcommand{\distrSym}{\OB{\delta}}
\newcommand{\distr}[1]{\OB{\distrSym(#1)}}


%%% Java example
% \newcommand{\etc}{\OB{\cdots}}
% \newcommand{\etc}{}
\newcommand{\tok}[1]{\texttt{\small #1}}
\newcommand{\toktr}[2]{\tr[\tok{#1}]{#2}}
\newcommand{\Nesting}[2]{\begin{array}[b]{@{}l@{}}%
     \tok{#1}\langle\\ \indnt#2\end{array}}
\newcommand{\End}{\rangle}
\newcommand{\indnt}{\hspace*{1em}}

\newcommand{\showEx}[1]{\medskip\noindent\centerline{#1}\medskip\noindent}

\newcommand{\class}[1]{\toktr{class}{#1}}
% \newcommand{\classNest}[1]{\Nesting{class}{#1}}
\newcommand{\List}{\tok{List}}
\newcommand{\Listg}{\tok{List<Job>}}
\newcommand{\voidName}{void}
\newcommand{\void}[1]{\toktr{\voidName}{#1}}
\newcommand{\intt}{\tok{int}}
\newcommand{\Iter}{\tok{Iter}}
\newcommand{\while}[1]{\toktr{while}{#1}}
\newcommand{\forr}[1]{\toktr{for}{#1}}
\newcommand{\Job}{\tok{Job}}
\newcommand{\run}{\tok{j.run}}
\newcommand{\ifff}{\tok{if}}
\newcommand{\nocode}{\OB{\bullet}}

% alternative content-based naming
\renewcommand{\voidName}{runJob}
\renewcommand{\intt}{\tok{trialCount}}
\renewcommand{\ifff}{\tok{break}}

\newcommand{\prel}{\OB{{\cal R}}}

% class, List/List<Job>, runJobs, trialCount, Iter, for/while, Job, j.run, break





\newcommand{\set}[1]{\ensuremath{\{#1\}}}
\newcommand{\dom}{\textit{dom}}
\newcommand{\rng}{\textit{rng}}
\newcommand{\eset}{\ensuremath{\varnothing}}
\newcommand{\OR}{\ |\ }
\newcommand{\BB}{\ensuremath{\mathbb{B}}}
\newcommand{\TT}{\ensuremath{true}}
\newcommand{\FF}{\ensuremath{false}}

% General language structure
%
\newcommand{\prog}[1]{{\small\texttt{#1}}}
\newcommand{\bs}{\texttt{\symbol{92}}}

\newcommand{\lblFmt}[1]{\textrm{\textit{#1}}}
\newcommand{\chcPP}[3][D]{\chc[#1]{\prog{#2},\prog{#3}}}
\newcommand{\chcPPP}[4][D]{\chc[#1]{\prog{#2},\prog{#3},\prog{#4}}}

% dimensions and tags
\newcommand{\dimA}[2][a_1,a_2]{\DimIn[A]{#1}{#2}}
\newcommand{\chcA}[1]{\chc[A]{#1}}
\newcommand{\dcA}[2][a_1,a_2]{\dimA[#1]{\chcA{#2}}}

\newcommand{\dimB}[2][b_1,b_2]{\DimIn[B]{#1}{#2}}
\newcommand{\chcB}[1]{\chc[B]{#1}}
\newcommand{\dcB}[2][b_1,b_2]{\dimB[#1]{\chcB{#2}}}

\newcommand{\map}[2][\dec]{\OB{#1\mapsto#2}}

% \newcommand{\dimset}[1][D]{\OB{\bar{#1}}}
\newcommand{\dimset}[1][D]{\OB{#1^n}}
\newcommand{\xdimSym}{\textit{dim}}
\newcommand{\xdim}[1]{\OB{\xdimSym(#1)}}
% \newcommand{\xtag}[1]{\OB{\textit{tag}(#1)}}
\newcommand{\dimsSym}{\textit{dims}}
\newcommand{\dims}[1]{\OB{\dimsSym(#1)}}
% \newcommand{\tags}[2]{\OB{\textit{tags}_{#1}(#2)}}

\newcommand{\decstr}{\OB{{\cal D}}}

\newcommand{\qt}[1][\decstr]{\OB{Q_{#1}}}

\newcommand{\deprel}{\OB{R}}
\newcommand{\dep}[2]{\OB{#1\mathop\to#2}}
\newcommand{\dec}{\OB{\delta}}
\newcommand{\Dec}[1][\decstr]{\OB{\Delta_{#1}}}

% \newcommand{\tg}[2][e]{\OB{#2^{#1}}}
\newcommand{\tagv}[1]{\OB{\bar{#1}}}
\newcommand{\tagged}[2][\decstr]{\OB{\tau_{#1}(#2)}}
\newcommand{\variT}[2][\decstr]{\OB{V_{#1}(#2)}}
\newcommand{\vari}[2][\decstr]{\OB{V_{#1}(2^{#2})}}
\newcommand{\variv}[1]{\OB{\vec{#1}}}
\newcommand{\vv}{\variv{v}}
\newcommand{\vf}{\variv{f}}
\newcommand{\vS}{\variv{S}}

\newcommand{\sel}[2]{\OB{#1!#2}}
\newcommand{\indep}[2]{\OB{#1\nrightleftarrows#2}}

\newcommand{\anddSym}{\OB{\wedge}}
\newcommand{\ordSym}{\OB{\vee}}
\newcommand{\andd}[2]{\OB{#1\anddSym#2}}
\newcommand{\ord}[2]{\OB{#1\ordSym#2}}
\newcommand{\all}{\OB{\star}}

\newcommand{\splt}[2]{\OB{#1\curlywedgedownarrow#2}}
\newcommand{\fcup}{\OB{\mathrel{\vec{\cup}}}}
\newcommand{\fcap}{\OB{\mathrel{\vec{\cap}}}}
\newcommand{\fsqcup}{\OB{\mathrel{\vec{\sqcup}}}}
\newcommand{\fsqcap}{\OB{\mathrel{\vec{\sqcap}}}}

% \newcommand{\nlab}{\OB{\nu}}
% \newcommand{\elab}{\OB{\eta}}
\newcommand{\nlab}{\OB{N}}
\newcommand{\elab}{\OB{E}}
\newcommand{\tN}{\tagv{\nlab}}
\newcommand{\tE}{\tagv{\elab}}
\newcommand{\tG}{\tagv{G}}
\newcommand{\tp}{\tagv{p}}
\newcommand{\vN}{\variv{\nlab}}
\newcommand{\vE}{\variv{\elab}}
\newcommand{\vG}{\variv{G}}
% \newcommand{\vG}{\variv{G}}

\newcommand{\pth}{\textit{path}}
\newcommand{\tpth}{\tagv{\textit{path}}}

\newcommand{\strp}[1]{\OB{\lfloor#1\rfloor}}

% \newcommand{\semL}{\OB{[\![}}
% \newcommand{\semR}{\OB{]\!]}}
\newcommand{\semL}{\llbracket}
\newcommand{\semR}{\rrbracket}
\newcommand{\sem}[2][{}]{\OB{\semL#2\semR_{#1}}}
\newcommand{\xsem}[2][{}]{\OB{\semL#2\semR_{#1}}}

\newcommand{\inclSym}{\ensuremath{\sqsubseteq}}
\newcommand{\incl}[2]{\ensuremath{#1 \inclSym #2}}


\newcommand{\cons}[2]{\OB{#1\mathop:#2}}

% \newcommand{\qu}{\OB{\omega}}
% \newcommand{\qd}{\OB{\rho}}
% \newcommand{\qr}{\OB{\varphi}}
% \newcommand{\id}{\OB{\iota}}

\newcommand{\filt}[2]{\OB{\langle#1,#2\rangle}}
\newcommand{\qu}{\OB{\phi}}
\newcommand{\qd}{\OB{\alpha}}
\newcommand{\qr}{\OB{\beta}}
\newcommand{\id}{\OB{\iota}}

\newcommand{\myHeader}[1]{\textbf{#1}.}

%%
%%
\newcommand{\figscale}{0.6}
\newcommand{\varsheet}{VarSheet}
\newcommand{\EUSES}{EUSES}
\newcommand{\gds}{goal-directed selection}

\newcommand{\da}{\OB{r}}
\newcommand{\spos}{\OB{p}}
\newcommand{\f}{\OB{f}}
\newcommand{\add}{\OB{a}}
\newcommand{\gid}{\OB{id}}
\newcommand{\tg}{\OB{t}}
\newcommand{\tgs}{\OB{2^t}}
\newcommand{\F}{\OB{F}}
\newcommand{\POS}{\OB{P}}
\newcommand{\p}{\OB{p}}
\newcommand{\vsheet}{\OB{s}}
\newcommand{\VSheet}{\OB{S}}
\newcommand{\V}[1]{\OB{V~#1}}
\newcommand{\TCons}{\OB{\prog{T}}}
\newcommand{\mapname}[1]{\textit{#1}}
\newcommand{\dset}[1]{\{#1\}}
\newcommand{\gcell}[1]{\##1}
\newcommand{\ltText}{\OB{<_{r}}}
\newcommand{\eqText}{\OB{\equiv_{r}}}
\newcommand{\ltRel}[2]{#1 \ltText #2}
\newcommand{\eqRel}[2]{#1 \eqText #2}

\newcommand{\pos}{\OB{\pi}}
\newcommand{\var}{\OB{V}}
\newcommand{\fml}{\OB{\varphi}}

\newcommand{\htdim}{Housing\&Transportation}
\newcommand{\lpdim}{Loan Payment}

\newcommand{\A}{\OB{A}}
\newcommand{\natset}{\mathbb{N}}
\newcommand{\checked}{\text{\ding{51}}}
\newcommand{\unchecked}{\text{\ding{55}}}
\newcommand{\undecided}{\OB{\bullet}}


\pagestyle{plain}
\date{}

\begin{document}

\title{A Model for Representing Variational Spreadsheets
\thanks{This work is partially
supported by the Air Force Office of Scientific
Research under the grant FA9550-09-1-0229 and by the
National Science Foundation under the grant CCF-0917092.
}}

\author{
\IEEEauthorblockN{Martin Erwig}
\IEEEauthorblockA{Oregon State University\\
                  erwig@eecs.oregonstate.edu}
\and
\IEEEauthorblockN{Duc Le}
\IEEEauthorblockA{Oregon State University\\
                  ledu@eecs.oregonstate.edu}
\and
\IEEEauthorblockN{Eric Walkingshaw}
\IEEEauthorblockA{Oregon State University\\
                  walkiner@eecs.oregonstate.edu}
}

\maketitle

\begin{abstract}
TBD
\end{abstract}

\section{Introduction}
\label{sec:intro}

\NOTE{New motivation needed}

We first give a motivating example for \varsheet. Figure~\ref{fig:before} shows a conventional spending estimate of
a college student. Suppose the student is not happy with it,
they would adjust the costs of some categories based on available options. After some adjustments, the student ends up with the spreadsheet in Figure~\ref{fig:after}, for which they pay \$50 less.
In the updated spreadsheet, the housing cost is \$50 less since the rented house is further from campus, but that would increase
the cost of transportation. The student also decides to pay \$100 less on their loan.
Considering both versions, the student would probably go after the latter one, and by doing this, they lose
the chance to reduce another \$50. Had the student chosen the original housing option and kept Loan payment to be \$500, the monthly
cost would have been \$1450.

Figure~\ref{fig:selection} shows a possible user interface of \varsheet~that could support the student in their decision making process.
There are three variation points in the spreadsheet. The \mapname{\htdim} categories are related and thus grouped
into a red box with a dashed line separating the two available options. 
In \varsheet, \mapname{\htdim} is called a \emph{dimension}---a choice users have to make. A dimension contains a set of options, each being called a \emph{tag}.
\mapname{\htdim}'s tags are \mapname{Close} and \mapname{Far}.
The \mapname{\lpdim} dimension represents a different variation point with two tags \mapname{500} and \mapname{600}
 and thus is colored differently (green). 
The last variation point, the \mapname{Total} category, does not contain variational formulas by itself. The formula being
used is \prog{SUM(B2:B6)} and is non-variational, yet the cell inherits the variational structures of its referred cells and
hence contains four available options. This cell is therefore colored purple to indicate that it contains \emph{induced variation}. Showing all the four alternatives of 
\mapname{Total} gives the student an overview of all the different options they
have. Moreover, if the student selects the \$1450 alternative, \varsheet~will automatically make
decisions for \mapname{\htdim} and \mapname{\lpdim}, and displays those decisions and the resulting spreadsheet to them.
We call this feature \emph{\gds}, the process of selecting spreadsheet variants based on certain goals.

\begin{figure}
\centering
    \subfloat[Before] {
        \includegraphics[scale=\figscale]{img/before}
        \label{fig:before}
    }\hspace{2em}
    \subfloat[After] {
        \includegraphics[scale=\figscale]{img/after}
        \label{fig:after}
    }\hspace{2em}
    \subfloat[A Possible User Interface] {
        \includegraphics[scale=\figscale]{img/selection}
        \label{fig:selection}
    }
\caption{A Monthly Spending Spreadsheet}
\label{fig:mspending}
\end{figure}



The study of variational spreadsheets brings up several insights to current research on software variation.
While traditional variation mechanisms focus on either the syntax or semantics domain, spreadsheets' immediate semantics computation expands the application of variational constructs to both domains, enabling the realization of \gds. 
In the example above, Loan payment varies syntactically while Total varies semanticallly, and one could even define cells that vary on both domains.

Existing variational contructs mainly work on linear or tree structures, which localizes their scope of impact. 
Spreadsheets have a special two dimensional structure that makes localization hard to achieve.
For instance, in Figure~\ref{fig:before}, one could replace row 4 by the spreadsheet in Figure~\ref{fig:food} and
expects this variation introduction to be local. This action unfortunately has a global impact on the spreadsheet's structure and the addresses/values of several unrelated cells.
The Monthly Cost column has to be shifted to column C, while the Total column has to be shifted one row down. 
In our variational spreadsheet model, we provide mechanisms to localize the effect of structural changes.

\begin{figure}
\centering
\includegraphics[scale=\figscale]{img/food}
\caption{A Different Way to Represent the Food Category}
\label{fig:food}
\end{figure}

Lastly, by letting users actively define variation in spreadsheets, we remove the need for using spreadsheet diffing algorithm,
which can be imperfect and misleading at times.

% In following sections of the paper, we first discuss the background and related work in section~\ref{sec:background}. Then we
% begin describing the syntax of \varsheet~in section \ref{sec:syntax}, which is followed by section \ref{sec:semantics} about the language's semantics.
% Section \ref{sec:langprops} discusses several axiomatic rules that could be used to transformed,
% while section \ref{sec:eval} provides an evaluation of our language on the \EUSES~corpus \cite{Ii05theeuses}. Section \ref{sec:concl} concludes the paper.

\section{Background and Related Work}
\label{sec:background}

\NOTE{To be rewritten}

\NOTE{Talk about the prototype of the VLHCC 2011 paper}

Existing empirical research demonstrates the need for an effective approach to deal with spreadsheet variation.
Spreadsheet reusing is common, but users often have to choose from various options [citation needed].
Once a spreadsheet is chosen and several modifications have been made to it,
if users recognize that it was not the right one to begin with, they will have to start all over again on a
different one. This could happen for many times until users are happy with their choice. To mitigate this problem, \varsheet~
gives users the ability to modify multiple versions/spreadsheets at the same time on a single representation and
select a desired version later.
Another reason for spreadsheet variation is due to spreadsheet errors and debugging.
Spreadsheets contain errors \cite{Panko98whatwe}\cite{Powell2008128}, many of which are introduced in the process of
reusing and modifying existing spreadsheets. When debugging, users often need to show the
differences between multiple versions, so a framework for systematically managing changes is needed.

In the area of spreadsheet change support, existing tools can be classified into two big categories: change tracking tools and spreadsheet diffing tools.
One representative example of a change tracking tool is Microsoft Excel's change tracking feature, which provides users the ability to track spreadsheet edits
such as inserting rows, updating equations, etc. This tool is useful and effective in helping users
understand versioning information of spreadsheets but is not
without limitation. The entire variational spreadsheet is represented using only the time dimension. It is not possible
to group changes into categories or groups such that they can be undone or applied again. Another problem arises when two or more users copy and modify
the same original spreadsheet. When trying to merge the modified copies, it is unclear which change includes or excludes other changes.
For \varsheet, grouping changes could simply be resolved by changing dimension/choice names, which will be defined in the next sections.
Research and commercial tools for diffing spreadsheets are prevalent, including CC DiffEngineX \cite{DiffEngineX} and Synkronizer \cite{Synkronizer}.
These tools are effective in comparing spreadsheets and producing accurate results. However, they do not reveal the original purposes
of users' changes and do not provide a way to document those.

On a broader topic, there has been extensive research on the topic of representing and managing software variation. The two big pillars of this topic
is the compositional approach \cite{KA08}, which modularizes software product line features \cite{CN01} into individual folders and
describes variability at a higher level using feature models \cite{Bat05}, and the annotated approach \cite{KA08}, where variability is encoded
and represented inside source code. Since there are advantages and disadvantages for each approach, Erwig and Walkingshaw~\cite{EW11tosem}
designed the Choice Calculus to shorten the gap between them and to take advantage of the approaches' benefits. The Choice Calculus's design
is based on the idea that software variation should be done at both source code and higher
levels with not-too-restrictive and not-too-relaxed constraints, making it highly
applicable for tree-like structures. \varsheet~expands Choice Calculus's ideas of dimensions
and choices to the spatial, two dimensional structure of spreadsheets.

\section{Dimensions and Decisions}
\label{sec:dims}

A \emph{dimension} describes one way in which something varies.  For example,
\mapname{\htdim} in our example in Figure~\ref{fig:mspending} is one
dimension of variation.
%
A \emph{dimension definition} assigns a \emph{dimension name} to a non-empty
set of \emph{tags}, which correspond to the alternatives in that dimension.  A
dimension definition is written as $D=\set{t_1,\ldots,t_n}$, 
%
for example,
% the mode of transportation dimension can be represented as
$\htdim=\set{Close, Far}$.
% 
A \emph{qualified tag} is a tag prefixed by the name of the dimension it is
taken from, written $D.t_i$. Qualified tags are used to distinguish between
tags of the same name from different dimensions. Given a qualified tag $q=D.t$,
we can extract the dimension name  with the function $\xdim{q}=D$.
% and the tag with the functions $\xdim{q}=D$ and $\xtag{q}=t$.


A \emph{decision space} of \emph{degree} $n$ is given by a set of $n$
dimension definitions $\dimset = \set{D_1=T_1, \ldots, D_n=T_n}$ where $T_i$
is the set of tags for dimension $D_i$. 
%
We define the function $\dims{\dimset}=\set{D_1,\ldots,D_n}$ to return the set
of all dimension names in a decision space.
% , and $\tags{\dimset}{D_i}=T_i$ to return
% the tags of dimension $D_i$ in decision space $D^n$.
%
The \emph{tag universe} of decision space $D^n$, written $\qt[D^n]$, is the set
of all qualified tags in $D^n$, defined as 
$\qt[D^n]=\set{D.t\ |\ D\in\dims\dimset\wedge t\in D}$.


A \emph{decision} in a decision space \dimset\ is a set of
qualified tags $\dec\subseteq\qt[D^n]$ that contains at most one tag for each
dimension, that is, $q,q'\in\dec\implies q=q'\vee\xdim{q}\neq\xdim{q'}$.
%
We overload the function \dimsSym\ to also denote the dimension names of a
decisions, that is, $\dims{\dec}=\bigcup_{q\in\dec}\xdim{q}$.
% $\dims{\dec}=\bigcup_{q\in\dec}\xdim{q}$.
% 
A decision $\dec\subseteq\qt[D^n]$ is \emph{complete} if it contains
a qualified tag from every dimension in $D^n$, that is, if
$\dims{\dec}=\dims{D^n}$, otherwise it is \emph{partial}.

\section{Variational Spreadsheets}

A variational spreadsheet represents a family of related 
plain spreadsheets, each being called a \emph{variant}.
Figure~\ref{fig:before} and~\ref{fig:after} show two of the four variants of the monthly spending spreadsheet.
Each variant contains a subset of a universe set of \emph{variational cells}.

Variational cells encode information about which
variants the cells belong to, where they appear in the two-dimentional grid structure, and what types of content they store.
Each variational cell has an unique address $\add$ from the set $\A$. 
A function $\var:\A\to2^{\qt[D^n]}$ maps the address to a subset of the tag universe.
The purpose of $\var$ is to decide if a cell is selected after a decision is made.
Each cell also contains a \emph{formula} $\f\in\F$, which can be a value, an address reference, or a function on other formulas.
\[
\begin{array}{lcl@{\qquad}l}
\f \in \F & ::= & v  & \textit{values} \\
         & \OR & \add & \textit{identity references} \\
         & \OR & \OB{\omega} (\f, \ldots, \f) & \textit{functions}\\
\end{array}
\]
\noindent
In the above definition, $\F$ represents the set of all formulas, $v$ ranges over primitive values 
(integers, etc.), and $\omega$ stands for the set of all possible functions. 
We define a function $\fml:\A\to\F$ to map cell addresses to formulas and a function $\pos:\A\to\POS$ to generate \emph{relative positions} given cell addresses.
Relative positions $\p = (\natset, \natset)$ are stored as pairs of natural numbers and are used to aid a pretty
printing algorithm in computing cells' absolute positions on two-dimentional grids.
The first number in $\p$ represents relative vertical positions while the second represents horizontal positions.

A variational spreadsheet combines a decision space $\dimset$ and the universe set of variational cells and is defined by the tuple $(\dimset, \var, \fml, \pos)$.

In Figure~\ref{fig:idedsheet} we provide two variants of a variational spreadsheet containing a single dimension $\mapname{D} = \set{D.1, D.2}$.
Relative positions are shown on the top-right corner of each cell, and cell addresses are shown on the top-left corner.
The universe set of cells contains all the cells with addresses \gcell{1--10}.
We leave the contents of several cells blank as they are not important for our discussion.
The cell at address \gcell{2}'s formula is \prog{\#5 + 1}, yet the formula is pretty printed as \prog{C1 + 1} and
\prog{D1 + 1} since cell \gcell{5}'s grid position changes from one variant to another. 
Two different types of cells exist in Figure~\ref{fig:idedsheet}, \emph{non-variational} and \emph{variational} cells.
Non-variational cells are cells that appear in all variants whereas variational cells do not. 
Non-variational cells' tags are the tag universe set $\qt[D^n]$, and variational cells' tags are proper subsets of $\qt[D^n]$.
We provide the cells' tags below.

\[
\begin{array}{l@{\ : \ }l}
    \gcell{1}, \gcell{2}, \gcell{5}, \gcell{6} & \set{D.1, D.2} \\
    \gcell{3}, \gcell{4} & \dset{\mapname{D.1}} \\
    \gcell{7}, \gcell{8}, \gcell{9}, \gcell{10} & \dset{\mapname{D.2}} \\
\end{array}
\]

The pretty printing algorithm places the cell with the lowest horizontal and vertical positions at the position \prog{A1} and recursively adds cells to the grid based on the following conventions.

\begin{itemize}
    \item Cells with the same vertical position are on the same column. Cells with the same horizontal position are on the same row.
    \item For all pairs of cells $x$ and $y$, if $x$'s horizontal position is less than $y$'s, $x$'s row number has to be less than $y$'s.
    \item For all pairs of cells $x$ and $y$, if $x$'s vertical position is less than $y$'s, $x$'s column number has to be less than $y$'s.
\end{itemize}

\begin{figure*} [ht]
\centering
\subfloat[Variant 1]{
    \includegraphics[scale=\figscale]{tikz/v1}
    \label{fig:v1}
}
\subfloat[Variant 2]{
    \includegraphics[scale=\figscale]{tikz/v2}
    \label{fig:v2}
}
\caption{Two variants of a variational spreadsheet}
\label{fig:idedsheet}
\end{figure*}

\section{Semantics}
\label{sec:semantics}
\newcommand{\semP}{\sem[P]}

The pretty printing algorithm works on individual variants, which are selected by picking a subset of cells from the cell universe.
A natural question is how do we know which cell to pick. To answer this question,
in this section we describe \emph{variation semantics}, a mapping between complete decisions and spreadsheet variants.\footnote{Note that
we ignore the discussion about the semantics of individual variants since
they are basically the semantics of plain spreadsheets.}

The steps involved in computing variation semantics are: (1) generating all complete decisions,
(2) performing \emph{tag selection} on those decisions to label each cell with either checked (\checked) or crossed (\unchecked), and 
(3) collect all checked cells into variants.

\subsection*{Dimensions and Complete Decisions}
\newcommand{\dimColText}{\textit{dims}}
\newcommand{\dimCol}[1]{\OB{\dimColText(#1)}}
\newcommand{\tagsText}{\textit{tags}}
\newcommand{\tagsCell}[1]{\OB{\tagsText(#1)}}
\newcommand{\decisionsText}{\textit{decisions}}
\newcommand{\decisions}[1]{\OB{\decisionsText(#1)}}
\newcommand{\examplesheet}{\OB{sh}}

The set of complete decisions of a variational spreadsheet \vsheet~is defined below with $D^n$ being \vsheet's decision space.
\begin{align*}
\decisions{s} = \{\{D_1.t_1, \ldots, D_n.t_n\} \ |\ & \{D_1,\ldots,D_n\} = \dims{D^n}\\
        & \wedge t_i \in D_i\}
\end{align*}


\subsection*{Tag Selection}



% The tag selection procedure $\tsel[ts]{\vsheet}$~takes a complete decision \OB{ts} and goes through
% a variational spreadsheet \vsheet~to filter out all cells whose tags are subsets of \OB{ts} into a variant.

% The two tag sequences $\OB{ts_1}$ and $\OB{ts_2}$ are compatible
% when either $ts_1 \subseteq ts_2$ or $ts_2 \subseteq ts_1$.
% \[
% \tsel[ts]{\vsheet} = \{c\ |\ c \leftarrow \vsheet, \tagsCell{c} \subseteq ts\}
% \]
% \noindent
% Notice the \tagsText~operation, which selects a variational cell's tags .

% Variation semantics is then defined as
% \[
% \begin{array}{r@{\ =\ }l}
    % \vari{\vsheet} & \{(ts, \tsel[ts]{\vsheet})\ |\ ts \leftarrow \decisions{\vsheet}\} \\
% \end{array}
% \]

% For the spreadsheet in Figure~\ref{fig:idedsheet}, the decisions are \dset{\mapname{D.1}} and \dset{\mapname{D.2}}.
% Given each cell's tags, \dset{\mapname{D.1}} is mapped to
% the set of cells \{\gcell{1}, \gcell{2}, \gcell{3}, \gcell{4}, \gcell{5}, \gcell{6}\}, and \dset{\mapname{D.2}} is mapped to
% the set of cells \{\gcell{1}, \gcell{2}, \gcell{5}, \gcell{6}, \gcell{7}, \gcell{8}, \gcell{9}, \gcell{10}\}.
% These two sets then become inputs for the pretty printer to
% generate cell addresses.

% \section{Concrete Syntax}
% \label{sec:concsyntax}

% The concrete syntax has two different modes: the \emph{variation exploration} mode and the default \emph{individual variant} mode.
% In the variation exploration mode, users can expand several variation points to show all the available alternatives as
% in Figure~\ref{fig:selection}. Users can also select a certain alternative of a variation point and corresponding decisions 
% will be made automatically. The individual variant mode displays individual variants to users. 
% To see a variant, users make decisions using the dimension panel as in
% Figure~\ref{fig:dimension_panel}, which visualizes the relationship between dimensions and tags.
% All tags of a dimension are grouped into a radiobutton group.
% Users make decision for that dimension by selecting one of the radiobuttons.
% The idea of dimension panel is reused from our previous work on variational representation of source code \cite{le2011}.
% This work confirmed that showing individual variants and separating the configuration structures into a dimension panel improved program comprehension.

% Each dimension is mapped to a distinct color and all cells relating to that dimension is annotated with
% the same color.
% This matching is only applicable when a cell's tags has only one dimension.
% When a cell is associated with two or more dimensions, we could apply different visual cues to represent the relationship.
% One approach could be creating a color palette at one of the cell's corner for storing the colors of all associated dimensions. Another approach is using a special color for the cell to indicate its dependence on multiple dimensions
% and then when the cell is highlighted, all related dimensions are also highlighted in the dimension panel.

% \begin{figure}
% \centering
% \includegraphics[scale=\figscale]{img/dimension_panel}
% \caption{Dimension Panel}
% \label{fig:dimension_panel}
% \end{figure}

% Producing the concrete presentation of spreadsheet variants requires taking the output of the
% pretty printer and positioning cells at appropriate addresses.
% We require that each variant has a rectangular shape, which results in cases where one
% has to add \emph{filler cells} to fill up empty spaces.
% Table~\ref{tbl:fourvar} gives an example where filler cells are needed. 
% There are four variants of a variational spreadsheet with two dimensions $D = \{D_1, D_2\}$ and $D' = \{D'_1, D'_2\}$.
% The column and row headers represent the corresponding decision of each variant.
% For instance, the decision \dset{$D_1, D'_2$} corresponds to the lower left variant.
% Each cell's tags are listed below.
% \[
% \begin{array}{l@{\ : \ }l}
    % \gcell{1} & \dset{D_1} \\
    % \gcell{2} & \dset{{D'_1}} \\
    % \gcell{3}, \gcell{4} & \dset{D_1} \\
    % \gcell{5}, \gcell{6} & \dset{{D'_2}} \\
% \end{array}
% \]
% Cell \gcell{7} and \gcell{8} are filler cells, that is, they are not represented or stored in our model, yet
% they are needed to ensure variants' rectangular shapes. Cell \gcell{7} is added whenever $D'_2$ is chosen.
% For example, our semantics function maps the decision \dset{$D_1, D'_2$} to the set of three cells \{\gcell{1}, \gcell{5}, \gcell{6}\}.
% The pretty printer then generates the corresponding addresses for each of these cells.
% \[
% \begin{array}{l@{\ : \ }l}
    % \gcell{1} & \prog{A1} \\
    % \gcell{5} & \prog{B1} \\
    % \gcell{6} & \prog{B2} \\
% \end{array}
% \]
% Since address \prog{A2} is not occupied by any cell, it becomes a ``black hole'' and is filled with cell \gcell{7}.
% While cell \gcell{7} is added when a singular choice $D'_2$ is chosen, \gcell{8} requires two choices $D_2$ and $D'_2$ 
% to be added to corresponding variants.

% \NOTE{TODO: Get rid of relative postions from \gcell{7} and \gcell{8} in Table~\ref{tbl:fourvar}}

% \begin{table*}
% \centering
  % \begin{tabular}{|>{\centering\arraybackslash}m{2em}|>{\centering\arraybackslash}m{18em} >{\centering\arraybackslash}m{25em}|}
    % \hline
    % & $D_1$ & $D_2$ \\ \hline
    % $D'_1$ & \includegraphics[scale=\figscale]{tikz/2dims1} 
      % & \includegraphics[scale=\figscale]{tikz/2dims2} \\
      % $D'_2$ & \includegraphics[scale=\figscale]{tikz/2dims3}
      % & \includegraphics[scale=\figscale]{tikz/2dims4} \\ \hline
  % \end{tabular}
  % \caption{Four Variants of a Spreadsheet}\label{tbl:fourvar}
% \end{table*}


\section{Partial Tag Selection and Goal-Directed Selection}
\newcommand{\ptsel}[2][s]{\OB{{\lfloor#2\rfloor}^p_{#1}}}
\newcommand{\compatibleSym}{\OB{\sim}}
\newcommand{\compatible}[2]{\OB{#1\compatibleSym#2}}
\newcommand{\updateText}{\textit{update}}
\newcommand{\update}[2]{\updateText(#1, #2)}

Variation semantics works with complete decisions, which is applicable
when users could make decisions for all dimensions. In practice, 
users might not know all decisions in advance. They might make decisions
for some dimensions, observe the outcome, and then either undo or reiterate the process.
We introduce the notion of \emph{partial tag selection} to support users
in making partial decision. Partial tag selection takes a variational
spreadsheet and a decision as input and returns another variational spreadsheet. 
When the input is a complete decision, partial tag selection returns a variant.
For instance, when applying the incomplete decision \dset{\mapname{\htdim.Close}}
on the spreadsheet in Figure~\ref{fig:mspending}, we obtain a spreadsheet
with \mapname{\lpdim} as the only dimension and the costs of housing and transportation become non-variational.  
 
Formally, the partially tag selection operation $\ptsel[]{} : \VSheet \times 2^{\tg} \rightarrow \VSheet$ is defined as
\[
\ptsel[ts]{\vsheet} = \{\update{c}{ts}\ |\ c \leftarrow \vsheet, \compatible{\tagsCell{c}}{ts} \}
\]

\noindent
The \updateText~operation removes all tags in \OB{ts} from cell \OB{c}'s tags.
If $\tagsCell{c} \subseteq ts$, \OB{c}'s tags becomes empty and the cell becomes
non-variational. The \compatibleSym~relation returns true if two sets of tags do not have
conflicts in decisions, that is, for every shared dimension,
corresponding tags must be the same. 
For instance, \compatible{\dset{D_1.t_1, D_2.t_2}}{\dset{D_1.t_1}} holds while \compatible{\dset{D_1.t_1, D_2.t_2}}{\dset{D_1.{t_1'}, D_2.t_2}} does not.

\[
\compatible{ts_1}{ts_2} = \forall D, D.{t_1} \in ts_1 \wedge D.{t_2} \in ts_2 \rightarrow t_1 = t_2
\]

Goal-directed selection is defined as partial tag selection on variational spreadsheets. Users
begin by inspecting all possible values of a cell. In Figure~\ref{fig:selection},
the cell being inspected is \prog{B8} and has four possible values, each
corresponding to a partial decision (In this example specifically, the partial decisions happen to be the complete decisions). The alternative \OB{1450} corresponds
to the decision \mapname{\{\htdim.Close, \lpdim.500\}}, which could be applied to
 the spreadsheet to obtain a single variant.

We simplify our discussion by looking four variants of a simpler spreadsheet in Table~\ref{tbl:costcalc}...


\begin{table*}
\centering
  \begin{tabular}{|>{\centering\arraybackslash}m{2em}|>{\centering\arraybackslash}m{25em} >{\centering\arraybackslash}m{25em}|}
    \hline
    & $D_1$ & $D_2$ \\ \hline
    $D'_1$ & \includegraphics[scale=\figscale]{tikz/costs1}
      & \includegraphics[scale=\figscale]{tikz/costs2} \\
      $D'_2$ & \includegraphics[scale=\figscale]{tikz/costs3} %\fixedlabel{fig:2dims3}{tikz/2dims3} 
      & \includegraphics[scale=\figscale]{tikz/costs4} \\ \hline
  \end{tabular}
  \caption{Four Variants of the Receipt Spreadsheet}\label{tbl:costcalc}
\end{table*}



\NOTE{TODO: (0) Provide a simpler example to talk about goal directed selection and syntactic sugar, (1) Talk about syntactic sugar (address referencing) here to enable \gds, (2) Describe \gds~ in more details, (3) Syntactic revisit: how to show all variants of a cell at a certain address?}

\section{Language Properties}
\label{sec:langprops}

\begin{theorem}
% If \compatible{ts_1}{ts_2} and $ts_1 \cap ts_2 = \emptyset$,
% \[
% \tsel[ts_1 \cup ts_2]{\vsheet} = \ptsel[ts_2]{\ptsel[ts_1]{\vsheet}} = \ptsel[ts_1]{\ptsel[ts_2]{\vsheet}}
% \]
\end{theorem}

\begin{proof}
...
\end{proof}



\NOTE{TODO: Add three more theorems: (1) C-S equivalent (choice expansion), (2) Reduction of tags, (3) Dimension dependency theorem}

% \NOTE{Some laws are: (1) Orders does not matter when performing tag selection.
% (2) What about the composition of two variational spreadsheets?
% (3) What does the addressing scheme affect the representation?
% }

\section{Conclusion and Future Work}
\label{sec:concl}
TBD

\bibliographystyle{abbrv}
\bibliography{change}

\end{document} 